\documentclass[a4paper,12pt]{article}
\usepackage{hyperref}
\usepackage{amsmath}
\usepackage{geometry}
\usepackage{fancyhdr}
\usepackage{enumitem}
\usepackage{tocbibind} % To include "Table of Contents" in the index
\geometry{margin=1in}
\pagestyle{fancy}
\fancyhf{}
\fancyhead[L]{Lab Notebook - Team 1}
\fancyhead[R]{\thepage}

\title{\textbf{Lab Notebook}}
\author{\textbf{Team 1}}
\date{}

\begin{document}

\maketitle

\begin{center}
    \Large\textbf{Maulana Abul Kalam Azad University of Technology}\\
    \vspace{0.2cm}
    \large Software Tools and Techniques - Lab Notebook
\end{center}

\vspace{1cm}

\section*{Assignment Details}
\begin{itemize}[leftmargin=1.5cm]
    \item \textbf{Assignment:} Create a Git Repository Containing a Lab Notebook in LaTeX Format
    \item \textbf{Subject:} Software Tools and Techniques
    \item \textbf{Team No.:} 1
    \item \textbf{GitHub Repo Link:} \url{https://github.com/MazidNawaz/Group_1_Latex.git}
\end{itemize}

\vspace{1cm}

\section*{Team Members}
\begin{itemize}[leftmargin=1.5cm]
    \item \textbf{Member 1 (Lead):} 
    \begin{itemize}[leftmargin=1.5cm]
        \item \textbf{Name: Mazid Nawaz Ahmad} 
        \item \textbf{Reg No.: 233002410602} 
        \item \textbf{Course: Bsc. IT Data Science} 
        \item \textbf{GitHub Link:} \url{https://github.com/MazidNawaz}
    \end{itemize}

    \item \textbf{Member 2:} 
    \begin{itemize}[leftmargin=1.5cm]
        \item \textbf{Name: Ratul Mondal} 
        \item \textbf{Reg No.: 233002410593} 
        \item \textbf{Course: Bsc. It Data Science} 
        \item \textbf{GitHub Link:} \url{}
    \end{itemize}

    \item \textbf{Member 3:} 
    \begin{itemize}[leftmargin=1.5cm]
        \item \textbf{Name: Srizani Dutta} 
        \item \textbf{Reg No.: 233002410005} 
        \item \textbf{Course: Bsc. Forensic Science} 
        \item \textbf{GitHub Link:} \url{https://github.com/srizani04}
    \end{itemize}

    \item \textbf{Member 4:} 
    \begin{itemize}[leftmargin=1.5cm]
        \item \textbf{Name:} 
        \item \textbf{Reg No.:} 
        \item \textbf{Course:} 
        \item \textbf{GitHub Link:} \url{}
    \end{itemize}

    \item \textbf{Member 5:} 
    \begin{itemize}[leftmargin=1.5cm]
        \item \textbf{Name:} 
        \item \textbf{Reg No.:} 
        \item \textbf{Course:} 
        \item \textbf{GitHub Link:} \url{}
    \end{itemize}
\end{itemize}

\vspace{1cm}

\section*{Table of Contents}
\tableofcontents

\vspace{1cm}

\newpage

% Section for each lab entry
\section{Lab 1: Calculator Program using C}

\subsection{Objective}
The objective of this lab is to develop a basic calculator program using the C programming language. The calculator will perform simple arithmetic operations like addition, subtraction, multiplication, and division based on user input.

\subsection{Program Overview}
The calculator program is designed to:
\begin{itemize}
    \item Accept two numbers from the user.
    \item Prompt the user to select an arithmetic operation (Addition, Subtraction, Multiplication, Division).
    \item Perform the selected operation.
    \item Display the result of the operation to the user.
\end{itemize}

The program includes error handling to manage division by zero and other invalid inputs.

\subsection{Code Implementation}
The following is the C code for the calculator program:

\begin{verbatim}
#include <stdio.h>

int main() {
    char operator;
    double num1, num2, result;

    printf("Enter an operator (+, -, *, /): ");
    scanf("%c", &operator);

    printf("Enter two operands: ");
    scanf("%lf %lf", &num1, &num2);

    switch(operator) {
        case '+':
            result = num1 + num2;
            break;
        case '-':
            result = num1 - num2;
            break;
        case '*':
            result = num1 * num2;
            break;
        case '/':
            if (num2 != 0)
                result = num1 / num2;
            else {
                printf("Error! Division by zero.\n");
                return -1;
            }
            break;
        default:
            printf("Error! Operator is not correct\n");
            return -1;
    }

    printf("Result: %.2lf\n", result);
    return 0;
}
\end{verbatim}

\subsection{Compiling and Running the Program}
To compile and run the calculator program:
\begin{enumerate}
    \item Open a terminal or command prompt.
    \item Navigate to the directory where the C file is located.
    \item Compile the program using a C compiler (e.g., GCC):
    \begin{verbatim}
    gcc calculator.c -o calculator
    \end{verbatim}
    \item Run the compiled program:
    \begin{verbatim}
    ./calculator
    \end{verbatim}
\end{enumerate}

\subsection{Adding the Calculator Program to GitHub Repository}
To add this calculator program to a GitHub repository, follow these steps:

\subsubsection{Step 1: Initialize a Local Git Repository}
\begin{enumerate}
    \item Open the terminal and navigate to the directory where your \texttt{calculator.c} file is located.
    \item If you haven't already, initialize a Git repository in that directory:
    \begin{verbatim}
    git init
    \end{verbatim}
    This command creates a new Git repository in the current directory.
\end{enumerate}

\subsubsection{Step 2: Add the File to the Repository}
\begin{enumerate}
    \item Add the \texttt{calculator.c} file to the staging area:
    \begin{verbatim}
    git add calculator.c
    \end{verbatim}
    This command stages the file, indicating that you want to include it in the next commit.
\end{enumerate}

\subsubsection{Step 3: Commit the Changes}
\begin{enumerate}
    \item Commit the file to the repository with a meaningful message:
    \begin{verbatim}
    git commit -m "Add calculator program in C"
    \end{verbatim}
\end{enumerate}

\subsubsection{Step 4: Push the Changes to GitHub}
\begin{enumerate}
    \item Link your local repository to a remote GitHub repository:
    \begin{verbatim}
    git remote add origin https://github.com/yourusername/your-repo-name.git
    \end{verbatim}
    \item Push the changes to the GitHub repository:
    \begin{verbatim}
    git push -u origin master
    \end{verbatim}
\end{enumerate}

\subsubsection{Step 5: Verify the Upload}
\begin{enumerate}
    \item Go to your GitHub repository URL in a web browser.
    \item Verify that the \texttt{calculator.c} file is listed and accessible in the repository.
\end{enumerate}


\section{Lab 2: Symbol Mind Reading Java Application}
% Lab details go here

% Add more sections as needed

\end{document}
